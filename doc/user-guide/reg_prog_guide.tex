\section{Register Interface}

The UART module is controlled via an APB interface that maps configuration, status, and data registers for both the transmitter (TX) and receiver (RX) sections. In the updated design, the module uses internal dynamic FIFOs to buffer data and a dedicated baud–rate generator to compute the effective clocks–per–bit. This section describes the register map and details the functionality of each register.

\subsection{Register Map Summary}

The table below summarizes the register space of the Uart module. All addresses assume a 32–bit APB address space.
It should be noted that all registers are \textbf{Double Buffered} and are automatically synchronized. This means that settings can be configured without the need for software synchronization.

\renewcommand*{\arraystretch}{1.25}
\begingroup
\small
\rowcolors{2}{gray!30}{gray!10}
\arrayrulecolor{gray!80}
\begin{longtable}{|c|c|c|c|p{0.25\textwidth}|}
    \hline
    \rowcolor{gray}
    \textcolor{white}{\textbf{Address}} & \textcolor{white}{\textbf{Register Name}} & \textcolor{white}{\textbf{Type}} & \textcolor{white}{\textbf{Reset Value}} & \textcolor{white}{\textbf{Description}} \\ \hline
    \endfirsthead

    \hline
    \rowcolor{gray}
    \textcolor{white}{\textbf{Address}} & \textcolor{white}{\textbf{Name}} & \textcolor{white}{\textbf{Type}} & \textcolor{white}{\textbf{Reset Value}} & \textcolor{white}{\textbf{Description}} \\ \hline
    \endhead

    \hline
    \endfoot

    0x0 &
    TX\_LOAD &
    R/W &
    0 &
    When set to 1, all data in the TX fifo is sent out sequentially. It is automatically reset to 0 and does not have to be reset.
    \\ \hline

    0x4 &
    TX\_DATAIN &
    R/W & 0 &
    When set, the data is sent to the back of the TX FIFO. It is automatically reset to 0 and does not have to be reset.
    \\ \hline

    0x8 &
    TX\_BAUDRATE &
    R/W &
    115,200 &
    Controls the baud rate of the TX module. Is updated after TX\_UPDATEBAUD is asserted.
    \\ \hline

    0xC &
    TX\_CLOCKFREQ &
    R/W &
    25,000,000 &
    This does not control the clock frequency, it is used by the divider to configure the TX frequency and must match the module clock frequency. Is updated after TX\_UPDATEBAUD is asserted.
    \\ \hline

    0x10 &
    TX\_UPDATEBAUD &
    R/W &
    0 &
    This tells the TX module to apply the changes in TX\_BAUDRATE, and TX\_CLOCKFREQ. It can take up to 32 cycles to converge.
    \\ \hline

    0x14 &
    TX\_NUMOUTPUTBITSDB &
    R/W &
    8 &
    This controls the number of data bits in a TX transaction.
    \\ \hline


    0x18 &
    TX\_USEPARITYDB &
    R/W &
    0 &
    This controls whether to use a parity bit in a TX transaction.
    \\ \hline

    0x1C &
    TX\_PARITYODDDB &
    R/W &
    0 &
    This controls whether to use odd or even parity in a TX transaction.
    \\ \hline

    0x20 &
    TX\_ALMOSTEMPTYLEVEL &
    R/W &
    1 &
    This is the level of the TX FIFO that triggers the TX\_FIFOALMOSTEMPTY flag.
    \\ \hline

    0x24 &
    TX\_ALMOSTFULLLEVEL &
    R/W &
    bufferSize - 1 &
    This is the level of the TX FIFO that triggers the TX\_FIFOALMOSTFULL flag.
    \\ \hline

    0x28 &
    TX\_FIFOFULL &
    R &
    0 &
    This flag is set when the TX FIFO is full.
    \\ \hline

    0x2C &
    TX\_FIFOEMPTY &
    R &
    1 &
    This flag is set when the TX FIFO is empty.
    \\ \hline

    0x30 &
    TX\_FIFOALMOSTEMPTY &
    R &
    0 &
    This flag is set when the TX FIFO is at or below the TX\_ALMOSTEMPTYLEVEL.
    \\ \hline

    0x34 &
    TX\_FIFOALMOSTFULL &
    R &
    0 &
    This flag is set when the TX FIFO is at or above the TX\_ALMOSTFULLLEVEL.
    \\ \hline

    0x38 &
    RX\_DATA &
    R &
    0 &
    This is the data at the top of the RX FIFO. \textbf{When read, it is removed from the FIFO}, and the next item is then shown.
    \\ \hline

    0x3C &
    RX\_DATAPEEK &
    R &
    0 &
    This is the data at the top of the RX FIFO. This can be read without removing the data from the FIFO.
    \\ \hline

    0x40 &
    RX\_DATAAVAILABLE &
    R &
    0 &
    This flag is set when there is data in the RX FIFO. Reading the RX\_DATA and RX\_DATAPEEK registers while this is low will result in undefined behavior.
    \\ \hline

    0x44 &
    ERROR &
    R &
    0 &
    This is the error register. See the following section on the error format for a more detailed description.
    \\ \hline

    0x48 &
    CLEARERROR &
    R/W &
    0 &
    This resets all error flags in the ERROR register when asserted. It is automatically reset to 0 and does not have to be reset.
    \\ \hline

    0x4C &
    RX\_BAUDRATE &
    R/W &
    115,200 &
    Controls the baud rate of the RX module. Is updated after RX\_UPDATEBAUD is asserted.
    \\ \hline

    0x50 &
    RX\_CLOCKFREQ &
    R/W &
    25,000,000 &
    This does not control the clock frequency, it is used by the divider to configure the RX frequency and must match the module clock frequency. Is updated after RX\_UPDATEBAUD is asserted.
    \\ \hline

    0x54 &
    RX\_UPDATEBAUD &
    R/W &
    0 &
    This tells the RX module to apply the changes in RX\_BAUDRATE, and RX\_CLOCKFREQ. It can take up to 32 cycles to converge.
    \\ \hline

    0x58 &
    RX\_NUMOUTPUTBITSDB &
    R/W &
    8 &
    This controls the number of data bits in a RX transaction.
    \\ \hline

    0x5C &
    RX\_USEPARITYDB &
    R/W &
    0 &
    This controls whether to use a parity bit in a RX transaction.
    \\ \hline

    0x60 &
    RX\_PARITYODDDB &
    R/W &
    0 &
    This controls whether to use odd or even parity in a RX transaction.
    \\ \hline

    0x64 &
    RX\_ALMOSTEMPTYLEVEL &
    R/W &
    1 &
    This is the level of the RX FIFO that triggers the RX\_FIFOALMOSTEMPTY flag.
    \\ \hline

    0x68 &
    RX\_ALMOSTFULLLEVEL &
    R/W &
    bufferSize - 1 &
    This is the level of the RX FIFO that triggers the RX\_FIFOALMOSTFULL flag.
    \\ \hline

    0x6C &
    RX\_FIFOFULL &
    R &
    0 &
    This flag is set when the RX FIFO is full.
    \\ \hline

    0x70 &
    RX\_FIFOEMPTY &
    R &
    0 &
    This flag is set when the RX FIFO is empty.
    \\ \hline

    0x74 &
    RX\_FIFOALMOSTEMPTY &
    R &
    0 &
    This flag is set when the RX FIFO is at or below the RX\_ALMOSTEMPTYLEVEL.
    \\ \hline

    0x78 &
    RX\_FIFOALMOSTFULL &
    R &
    0 &
    This flag is set when the RX FIFO is at or above the RX\_ALMOSTFULLLEVEL.
    \\ \hline

    0x7C &
    RX\_CLOCKSPERBIT &
    R &
    0 &
    This is the calculated number of clock cycles per bit in the RX module.
    \\ \hline

    0x80 &
    TX\_CLOCKSPERBIT &
    R &
    0 &
    This is the calculated number of clock cycles per bit in the TX module.
    \\ \hline

    0x84 &
    RX\_LSBFIRST &
    R/W &
    1 &
    This controls whether the RX module expects data LSB first or MSB first.
    \\ \hline

    0x88 &
    TX\_LSBFIRST &
    R/W &
    1 &
    This controls whether the TX module sends data LSB first or MSB first.
    \\ \hline

    0x8C &
    RX\_FLUSH &
    R/W &
    0 &
    When asserted, this tells the RX module to flush the RX FIFO. It is automatically reset to 0 and does not have to be reset.
    \\ \hline

    0x90 &
    TX\_FLUSH &
    R/W &
    0 &
    When asserted, this tells the TX module to flush the TX FIFO. It is automatically reset to 0 and does not have to be reset.
    \\ \hline
\end{longtable}
\captionof{table}{UART Register Map}
\label{table:uart_register_map}
\endgroup

\subsection{Error Register Description}

The error register is a 8-bit register that contains the following error flags:
\renewcommand*{\arraystretch}{1.25}
\begingroup
\small
\rowcolors{2}{gray!30}{gray!10}
\arrayrulecolor{gray!80}
\begin{longtable}{|c|c|c|c|c|c|c|c|}
    \hline
    \rowcolor{gray}
    \textcolor{white}{Bit 7} & \textcolor{white}{Bit 6} & \textcolor{white}{Bit 5} & \textcolor{white}{Bit 4} & \textcolor{white}{Bit 3} & \textcolor{white}{Bit 2} & \textcolor{white}{Bit 1} & \textcolor{white}{Bit 0} \\ \hline
    \endfirsthead

    \hline
    \rowcolor{gray}
    \textcolor{white}{Bit 7} & \textcolor{white}{Bit 6} & \textcolor{white}{Bit 5} & \textcolor{white}{Bit 4} & \textcolor{white}{Bit 3} & \textcolor{white}{Bit 2} & \textcolor{white}{Bit 1} & \textcolor{white}{Bit 0} \\ \hline
    \endhead

    \hline
    \endfoot

    TOP1 & RX3 & RX2 & RX1 & TX3 & TX2 & TX1 & ADDR1 \\ \hline
\end{longtable}
\captionof{table}{UART Error Description}
\label{table:uart_error}
\endgroup

\subsubsection{Top Error Register Bits}
\renewcommand*{\arraystretch}{1.25}
\begingroup
\small
\rowcolors{2}{gray!30}{gray!10}
\arrayrulecolor{gray!80}
\begin{longtable}{|c|c|}
    \hline
    \rowcolor{gray}
    \textcolor{white}{Value} & \textcolor{white}{Description}  \\ \hline
    \endfirsthead

    \hline
    \rowcolor{gray}
    \textcolor{white}{Value} & \textcolor{white}{Description}  \\ \hline
    \endhead

    \hline
    \endfoot

    0 & No Error \\ \hline
    1 & Invalid Register Programming \\ \hline
\end{longtable}
\captionof{table}{UART Top Error Description}
\label{table:uart_top_error}
\endgroup

\subsubsection{RX Error Register Bits}
\renewcommand*{\arraystretch}{1.25}
\begingroup
\small
\rowcolors{2}{gray!30}{gray!10}
\arrayrulecolor{gray!80}
\begin{longtable}{|c|c|}
    \hline
    \rowcolor{gray}
    \textcolor{white}{Value} & \textcolor{white}{Description}  \\ \hline
    \endfirsthead

    \hline
    \rowcolor{gray}
    \textcolor{white}{Value} & \textcolor{white}{Description}  \\ \hline
    \endhead

    \hline
    \endfoot

    0 & No Error \\ \hline
    1 & Start Bit Error \\ \hline
    2 & No Stop Bit was Detected \\ \hline
    3 & A Parity Mismatch was Detected \\ \hline
    4 & The FIFO was read while empty \\ \hline
    5 & The FIFO overflowed \\ \hline
\end{longtable}
\captionof{table}{UART RX Error Description}
\label{table:uart_rx_error}
\endgroup

\subsubsection{TX Error Register Bits}
\renewcommand*{\arraystretch}{1.25}
\begingroup
\small
\rowcolors{2}{gray!30}{gray!10}
\arrayrulecolor{gray!80}
\begin{longtable}{|c|c|}
    \hline
    \rowcolor{gray}
    \textcolor{white}{Value} & \textcolor{white}{Description}  \\ \hline
    \endfirsthead

    \hline
    \rowcolor{gray}
    \textcolor{white}{Value} & \textcolor{white}{Description}  \\ \hline
    \endhead

    \hline
    \endfoot

    0 & No Error \\ \hline
    1 & A Parity Mismatch was Detected \\ \hline
    2 & No Stop Bit was Detected \\ \hline
    3 & The FIFO was loaded while empty \\ \hline
    4 & The FIFO overflowed \\ \hline
\end{longtable}
\captionof{table}{UART TX Error Description}
\label{table:uart_tx_error}
\endgroup

\subsubsection{Addr Decoder Error Register Bits}
\renewcommand*{\arraystretch}{1.25}
\begingroup
\small
\rowcolors{2}{gray!30}{gray!10}
\arrayrulecolor{gray!80}
\begin{longtable}{|c|c|}
    \hline
    \rowcolor{gray}
    \textcolor{white}{Value} & \textcolor{white}{Description}  \\ \hline
    \endfirsthead

    \hline
    \rowcolor{gray}
    \textcolor{white}{Value} & \textcolor{white}{Description}  \\ \hline
    \endhead

    \hline
    \endfoot

    0 & No Error \\ \hline
    1 & An invalid address was detected on the APB Interface \\ \hline
\end{longtable}
\captionof{table}{UART Addr Decoder Error Description}
\label{table:uart_addrdecode_error}
\endgroup

\subsection{Parameter Relationships and Calculations}

\textbf{Key Parameter Relationships:}
\begin{itemize}
    \item \texttt{bufferSize}: This controls the default value of the TX\_ALMOSTFULLLEVEL and RX\_ALMOSTFULLLEVEL registers.
\end{itemize}

\subsection{Programming Template}

Below is an example header file and source file to use the Uart module.

\subsubsection{Header File}

\begin{lstlisting}[language=C,frame=single,label={lst:header}]

#ifndef REGISTER_MAP_H
#define REGISTER_MAP_H

#define TX_LOAD_OFFSET 0x0
#define TX_DATAIN_OFFSET 0x4
#define TX_BAUDRATE_OFFSET 0x8
#define TX_CLOCKFREQ_OFFSET 0xC
#define TX_UPDATEBAUD_OFFSET 0x10
#define TX_NUMOUTPUTBITSDB_OFFSET 0x14
#define TX_USEPARITYDB_OFFSET 0x18
#define TX_PARITYODDDB_OFFSET 0x1C
#define TX_ALMOSTEMPTYLEVEL_OFFSET 0x20
#define TX_ALMOSTFULLLEVEL_OFFSET 0x24
#define TX_FIFOFULL_OFFSET 0x28
#define TX_FIFOEMPTY_OFFSET 0x2C
#define TX_FIFOALMOSTEMPTY_OFFSET 0x30
#define TX_FIFOALMOSTFULL_OFFSET 0x34
#define RX_DATA_OFFSET 0x38
#define RX_DATAPEEK_OFFSET 0x3C
#define RX_DATAAVAILABLE_OFFSET 0x40
#define ERROR_OFFSET 0x44
#define CLEARERROR_OFFSET 0x48
#define RX_BAUDRATE_OFFSET 0x4C
#define RX_CLOCKFREQ_OFFSET 0x50
#define RX_UPDATEBAUD_OFFSET 0x54
#define RX_NUMOUTPUTBITSDB_OFFSET 0x58
#define RX_USEPARITYDB_OFFSET 0x5C
#define RX_PARITYODDDB_OFFSET 0x60
#define RX_ALMOSTEMPTYLEVEL_OFFSET 0x64
#define RX_ALMOSTFULLLEVEL_OFFSET 0x68
#define RX_FIFOFULL_OFFSET 0x6C
#define RX_FIFOEMPTY_OFFSET 0x70
#define RX_FIFOALMOSTEMPTY_OFFSET 0x74
#define RX_FIFOALMOSTFULL_OFFSET 0x78
#define RX_CLOCKSPERBIT_OFFSET 0x7C
#define TX_CLOCKSPERBIT_OFFSET 0x80
#define RX_LSBFIRST_OFFSET 0x84
#define TX_LSBFIRST_OFFSET 0x88
#define RX_FLUSH_OFFSET 0x8C
#define TX_FLUSH_OFFSET 0x90

#endif REGISTER_MAP_H
\end{lstlisting}

\subsubsection{TX Source File}
\begin{lstlisting}[language=C,frame=single,label={lst:tx-source}]

#include <stdio.h>
#include <stdlib.h>
#include <fcntl.h>
#include <unistd.h>
#include <sys/mman.h>
#include <stdint.h>
#include "uartRegs.h"

// Base address for the UART registers and mapping size
#define UART_BASE 0x43C20000
#define REG_SIZE  0x1000

// Transmitter register offsets (32-bit registers)

// Hypothetical TX FIFO empty flag register offset.
// Assumed to return 1 when the FIFO is empty.
//#define TX_FIFO_EMPTY_OFFSET          0x60

int main(void)
{
    int fd = open("/dev/mem", O_RDWR | O_SYNC);
    if(fd < 0) {
        perror("Failed to open /dev/mem");
        return -1;
    }

    volatile uint8_t *base = mmap(
        NULL,
        REG_SIZE,
        PROT_READ | PROT_WRITE,
        MAP_SHARED,
        fd,
        UART_BASE
    );
    if(base == MAP_FAILED) {
        perror("Failed to mmap UART registers");
        close(fd);
        return -1;
    }

    // 1. Configure TX baud rate and data format
    // Set TX system clock frequency to 25 MHz
    *(volatile uint32_t *)(base + TX_CLOCKFREQ_OFFSET) = 50000000;

    // Set desired baud rate to 115200
    *(volatile uint32_t *)(base + TX_BAUDRATE_OFFSET) = 115200;

    // Trigger TX baud rate update
    *(volatile uint32_t *)(base + TX_UPDATEBAUD_OFFSET) = 1;

    // Configure for 8 data bits per frame
    *(volatile uint32_t *)(base + TX_NUMOUTPUTBITSDB_OFFSET) = 8;

    // Disable parity
    *(volatile uint32_t *)(base + TX_USEPARITYDB_OFFSET) = 0;

    // (Ignored if parity is disabled)
    *(volatile uint32_t *)(base + TX_PARITYODDDB_OFFSET) = 0;

    // 2. Enqueue data for transmission
    // Write data word (0x55)
    *(volatile uint32_t *)(base + TX_DATAIN_OFFSET) = 0x33;

    // Pulse 'load' to push data into TX FIFO
    *(volatile uint32_t *)(base + TX_LOAD_OFFSET)    = 1;

    printf("Data transmitted successfully.\n");

    munmap((void *)base, REG_SIZE);
    close(fd);
    return 0;
}
\end{lstlisting}

\subsubsection{RX Source File}
\begin{lstlisting}[language=C,frame=single,label={lst:rx-source}]
#include <stdio.h>
#include <stdlib.h>
#include <fcntl.h>
#include <unistd.h>
#include <sys/mman.h>
#include <stdint.h>
#include "uartRegs.h"

#define UART_BASE 0x43C30000
#define REG_SIZE  0x1000

// Receiver register offsets (32-bit registers)
// Hypothetical register to indicate that data has been read (pop the RX FIFO)
// Not in the original map but referenced by the pseudocode.

int main(void)
{
    int fd = open("/dev/mem", O_RDWR | O_SYNC);
    if(fd < 0) {
        perror("Failed to open /dev/mem");
        return -1;
    }

    volatile uint8_t *base = mmap(
        NULL,
        REG_SIZE,
        PROT_READ | PROT_WRITE,
        MAP_SHARED,
        fd,
        UART_BASE
    );
    if(base == MAP_FAILED) {
        perror("Failed to mmap UART registers");
        close(fd);
        return -1;
    }

    // 1. Configure RX baud rate and data format
    // Set RX system clock frequency to 25 MHz
    *(volatile uint32_t *)(base + RX_CLOCKFREQ_OFFSET) = 50000000;

    // Set desired baud rate to 115200
    *(volatile uint32_t *)(base + RX_BAUDRATE_OFFSET) = 115200;

    // Trigger RX baud rate update
    *(volatile uint32_t *)(base + RX_UPDATEBAUD_OFFSET) = 1;

    // Configure for 8 data bits per frame
    *(volatile uint32_t *)(base + RX_NUMOUTPUTBITSDB_OFFSET) = 8;

    // Disable parity
    *(volatile uint32_t *)(base + RX_USEPARITYDB_OFFSET) = 0;

    // (Ignored if parity is disabled)
    *(volatile uint32_t *)(base + RX_PARITYODDDB_OFFSET) = 0;

    // 2. Poll for data reception: wait until RX FIFO is not empty.
    // (Assuming RX_DATA_AVAILABLE returns a nonzero value when data is available.)
    while(*(volatile uint32_t *)(base + RX_DATAAVAILABLE_OFFSET) == 0) {
        usleep(1000);        // Wait for data to be received
    }

    // 3. Read the received data word
    uint32_t rx_data = *(volatile uint32_t *)(base + RX_DATA_OFFSET);
    printf("Received data: 0x%02X\n", rx_data);

    // 5. Check for any reception errors
    uint32_t err = *(volatile uint32_t *)(base + TOP_ERROR_OFFSET);
    if(err != 0) {
        printf("UART RX error: 0x%02X. Clearing error...\n", err);
        *(volatile uint32_t *)(base + CLEARERROR_OFFSET) = 1;
    }

    munmap((void *)base, REG_SIZE);
    close(fd);
    return 0;
}
\end{lstlisting}



