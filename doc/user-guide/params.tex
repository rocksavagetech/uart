\section{Parameter Descriptions}
The parameters for \textbf{UART} are shown below in Table~\ref{table:params}.

\renewcommand*{\arraystretch}{1.4}
\begin{longtable}[H]{
    | p{0.25\textwidth}
    | p{0.10\textwidth}
    | p{0.05\textwidth}
    | p{0.05\textwidth}
    | p{0.47\textwidth} |
  }
  \hline
  \textbf{Name} &
  \textbf{Type} &
  \textbf{Min}  &
  \textbf{Max}  &
  \textbf{Description}            \\ \hline \hline
  dataWidth &
  Int &
  8 &
  64 &
  Width of the APB data bus in bits \\ \hline
  addressWidth &
  Int &
  8 &
  64 &
  Width of the APB address bus in bits \\ \hline
  maxOutputBits &
  Int &
  5 &
  9 &
  Maximum number of data bits per frame \\ \hline
  syncDepth &
  Int &
  1 &
  4 &
  Number of synchronization stages for RX input \\ \hline
  maxBaudRate &
  Int &
  9600 &
  921600 &
  Maximum supported baud rate \\ \hline
  maxClockFrequency &
  Int &
  1000000 &
  100000000 &
  Maximum system clock frequency \\ \hline
  parity &
  Boolean &
  - &
  - &
  Default parity enable state \\ \hline
  verbose &
  Boolean &
  - &
  - &
  Enable verbose debug output \\ \hline
  
  \caption{Parameter Descriptions}\label{table:params}
\end{longtable}

The UART is instantiated as follows:
\begin{lstlisting}[language=Scala]
val uartParams = UartParams(
  dataWidth = 32,
  addressWidth = 32,
  maxOutputBits = 8,
  syncDepth = 2,
  maxBaudRate = 921600,
  maxClockFrequency = 25000000
)
val uart = new Uart(uartParams)
\end{lstlisting}