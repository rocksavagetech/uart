\section{Parameter Descriptions}

The parameters for the \textbf{UART} module are listed in Table~\ref{table:uart_params}. These parameters configure the APB interface, baud rate generation, data framing, and the internal FIFO buffering for both TX and RX.

% \begin{itemize}
%     \item \texttt{dataWidth}: Width of the APB data bus (e.g., 32 bits).
%     \item \texttt{addressWidth}: Width of the APB address bus (e.g., 32 bits).
%     \item \texttt{wordWidth}: Width of the data word, usually 8 bits / 1 byte.
%     \item \texttt{maxOutputBits}: Maximum number of data bits in a frame.
%     \item \texttt{syncDepth}: Depth of the double-buffered registers.
%     \item \texttt{maxBaudRate}: Maximum baud rate supported by the UART.
%     \item \texttt{maxClockFrequency}: Maximum clock frequency of the UART).
%     \item \texttt{bufferSize}: Depth of the internal FIFOs.
% \end{itemize}

\renewcommand*{\arraystretch}{1.3}
\begingroup
\small
\rowcolors{2}{gray!30}{gray!10}
\arrayrulecolor{gray!80}

\begin{longtable}[H]{
  | p{0.22\textwidth}
  | p{0.13\textwidth}
  | p{0.08\textwidth}
  | p{0.08\textwidth}
  | p{0.42\textwidth} |
}
\hline
\rowcolor{gray}
\textcolor{white}{\textbf{Name}} &
\textcolor{white}{\textbf{Type}} &
\textcolor{white}{\textbf{Min}} &
\textcolor{white}{\textbf{Max}} &
\textcolor{white}{\textbf{Description}} \\ 
\hline
\endfirsthead

\hline
\rowcolor{gray}
\textcolor{white}{\textbf{Name}} &
\textcolor{white}{\textbf{Type}} &
\textcolor{white}{\textbf{Min}} &
\textcolor{white}{\textbf{Max}} &
\textcolor{white}{\textbf{Description}} \\ 
\hline
\endhead

\hline
\endfoot

\texttt{dataWidth} &
Integer &
8 &
32 &
Data width for APB bus operations. \\ \hline

\texttt{addressWidth} &
Integer &
8 &
32 &
Address width for APB bus. \\ \hline

\texttt{wordWidth} &
Integer &
8 &
8 &
Width of the data word (e.g., 8 bits). \\ \hline

\texttt{maxClockFrequency} &
Integer &
-- &
-- &
Maximum system clock frequency in Hz (used by the internal Baud Rate Generator). \\ \hline

\texttt{maxBaudRate} &
Integer &
-- &
-- &
Maximum achievable baud rate (used to size internal counters). \\ \hline

\texttt{maxOutputBits} &
Integer &
5 &
16 &
Maximum number of data bits plus optional parity bit (TX/RX). \\ \hline

\texttt{syncDepth} &
Integer &
2 &
-- &
Depth for RX input synchronization flip‐flops. \\ \hline

\textbf{\texttt{bufferSize}} &
Integer &
1 &
-- &
Depth of the \textit{DynamicFifo} used in TX and RX. Must be a power of 2 for certain synthesis flows. \\ \hline

\texttt{verbose} &
Bool &
N/A &
N/A &
Enables debug \texttt{printf} statements. \\ \hline

\texttt{coverage} &
Bool &
N/A &
N/A &
Enables collection of port toggle collection information. \\ \hline

\end{longtable}
\captionof{table}{UART Parameter Descriptions}
\label{table:uart_params}
\endgroup


\textbf{Example Instantiation:}

A typical instantiation of the UART module in Scala might be as follows:

\begin{lstlisting}[language=Scala]
val myUart = Module(new Uart(
  UartParams(
    dataWidth = 32,
    addressWidth = 32,
    wordWidth = 8,
    maxOutputBits = 8,
    syncDepth = 2,
    maxBaudRate = 25_000_000,
    maxClockFrequency = 25_000_000,
    coverage = true,
    verbose = true,
  )
  formal = false
))
\end{lstlisting}